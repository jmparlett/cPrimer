\documentclass{beamer}
\usepackage[utf8]{inputenc}
\usepackage[english]{babel}
\usepackage{amsmath}
\usepackage{amssymb}
\usepackage{fancyhdr}
\usepackage{pgfplots}
\usepackage{setspace}
\usepackage{listings}
\pgfplotsset{compat=1.17} 
\usepackage{enumerate}
\usepackage{algorithm}
\usepackage{algpseudocode}
% \geometry{a4paper} % or letter or a5paper or ... etc
% \geometry{landscape} % rotated page geometry
% \usepackage[margin=2cm]{geometry}
\usepackage{minted}
\usepackage[most]{tcolorbox}
\newtcolorbox{tb}[1][]{%
  sharp corners,
  enhanced,
  colback=white,
  height=6cm,
  attach title to upper,
  #1
}

%These setting will make the code areas look Pretty
\lstset{
	escapechar=~,
	numbers=left, 
	%numberstyle=\tiny, 
	stepnumber=1, 
	firstnumber=1,
	%numbersep=5pt,
	language=C,
	% stringstyle=\itfamily,
	%basicstyle=\footnotesize, 
	showstringspaces=false,
	frame=single,
  upquote=true
}

% created 2022-May-23 %
% Theme choice:
% \usetheme{AnnArbor}
\usetheme{focus}
% Title page details: 
\title{Memory and Pointers in C}
\author{Jonathan Parlett}
\date{\today}

\begin{document}

% Title page frame
\begin{frame}
    \titlepage
	{\bf Dynamic Array Programming Example}
\end{frame}

\begin{frame}{What were going to do}
	\begin{itemize}[<+->]
		\item Using what we've covered so far we are going to create a dynamic data structure.
		\item We will create a dynamic array that grows in size automatically as we store elements in it.
		\item We will use a pointer to store our data. 
		\item We will adjust the amount of memory allocated to this pointer in our program as the array grows in size.
	\end{itemize}
\end{frame}

\begin{frame}{Python Lists}
	\begin{itemize}[<+->]
		\item Python lists are an amazing data structure.
		\item We can store an arbitrary amount of elements and even store elements of different types in the same list.
		\item Knowing what you know about memory how do you think they store so many elements?
		\item It would be very inefficient to simply allocate space for a billion items and hope no one goes over the limit.
		\item Python lists grow in size as items are added to them.
		\item The underlying data structure is in fact an array.
		\item The size of this array is what is adjusted as items are added.
	\end{itemize}
\end{frame}
\begin{frame}{Things to keep in mind}
	\begin{itemize}[<+->]
		\item Try to relate what we are doing to the concepts we've previously discussed.
		\item Dynamic vs Static memory. As we write try to notice where in the program we are using dynamic, and static memory.
		\item Stack and Heap. As we write the program try to think about what information we are storing on the Heap and what we are storing on the stack.
		\item Memory model. If a structure looks confusing try to use the model we showed previously to visualize how it might be stored.
		\item Everyone has their own style of writing and thinking. This is just mine. You don't have to do things exactly as I do them.
		\item Lets get started.
	\end{itemize}
\end{frame}

\end{document}
